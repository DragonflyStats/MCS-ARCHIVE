\documentclass[a4]{beamer}
\usepackage{amssymb}
\usepackage{graphicx}
\usepackage{subfigure}
\usepackage{newlfont}
\usepackage{amsmath,amsthm,amsfonts}
%\usepackage{beamerthemesplit}
\usepackage{pgf,pgfarrows,pgfnodes,pgfautomata,pgfheaps,pgfshade}
\usepackage{mathptmx}  % Font Family
\usepackage{helvet}   % Font Family
\usepackage{color}

\mode<presentation> {
 \usetheme{Frankfurt} % was CambridgeUS
 \useinnertheme{rounded}
 \useoutertheme{infolines}
 \usefonttheme{serif}
 %\usecolortheme{wolverine}
% \usecolortheme{rose}
\usefonttheme{structurebold}
}

\setbeamercovered{dynamic}

\title[MCS]{Diagnostic measures for the analysis of method comparison
studies using the linear mixed effects model
 \\ {\normalsize Transfer Report Presentation}}
\author[Kevin O'Brien]{Kevin O'Brien \\ {\scriptsize kevin.obrien@ul.ie}}
\date{November 2010}
\institute[Maths \& Stats]{Dept. of Mathematics \& Statistics, \\ University \textit{of} Limerick}

\renewcommand{\arraystretch}{1.5}

\begin{document}

\begin{frame}
\titlepage
\end{frame}

\section{Method Comparison Studies}
\frame{

Method comparison studies:
\vspace{0.2cm}
\begin{itemize}
  \item assesses the level of agreement between two methods;

\end{itemize}
Very important type of analysis in biomedical sciences.\\
\vspace{0.2cm}
Agreement between two methods of clinical measurement can be quantified using the differences between observations made using the two methods on the same subjects (Bland and Altman 1999).\\
\vspace{0.2cm}
Inter-method bias\\


}
%----------------------------------------------------------------------------------------------%
\frame{\frametitle{Bland-Altman Plot}

The Bland-Altman plot is a plot of the casewise differences and averages of each pair of measurements.\\
\vspace{0.2cm}
It is easy to construct, and has become prevalent in medical literature.
\vspace{0.2cm}
Disadvantages:
\vspace{0.2cm}
It is constructed for a single pair of measurements.

}

% Presentation for Transfer
%------------------------------------------------
% Section 1: Introduction to MCS, BA plot, LoA
% Section 2: Roy's Methodology
% Section 3: Diagnostics











%----------------------------------------------------------------------------------------------%
\begin{frame}
\frametitle{Multidisciplinary Research Groups and Themes}
  \begin{columns}[t]
    \column{.4\textwidth}
    \begin{exampleblock}{Research Groups}<2->
      \begin{itemize}
          \item Clinical Trails and Health Care Evaluation
          \item<3-> \textbf{Q}uantifying \textbf{U}ncertainty in \textbf{E}pidemiological \textbf{S}tudies and \textbf{T}rials
          \item<4-> Applied, Bayesian and Computational Statistics
          \item[]
      \end{itemize}
    \end{exampleblock}
    \column{.5\textwidth}
    \begin{exampleblock}{Postgraduate and Post-Doctorial Scholars}<5->
     \begin{enumerate}
       \item<5-> Mathematics, Statistics, Engineering, Computer Science...
       \item[]<6-> \alert{Biology and Technology}
       \item<5-> Public Health, Epidemiology, Economics, Medicine, Midwifery...
       \item[]<6-> \alert{Biology and Technology}
       \item[]
     \end{enumerate}
    \end{exampleblock}
  \end{columns}
\end{frame}
%----------------------------------------------------------------------------------------------%
\end{document}

%----------------------------------------------------------------------------------------------%
\begin{frame}\frametitle{Diagnostics for LME models}
Anuradha Roy proposes a formal procedure for comparing two methods using LME Models.

\vspace{0.4cm}

Diagnostics methods have been developed for LME models. \\ \vspace{0.2cm}

A - Questions in this section are of ``Likert'' type. The data obtained here is ordinal (Categorical) although we treat it as if it were interval (Numerical) for the analysis.\\
\vspace{0.2cm}
B - One question asking people to indicate what School they are from - nominal (Categorical) data.\\
\vspace{0.2cm}
C - Another Likert question.
\end{frame}
%----------------------------------------------------------------------------------------------%
\frame{\frametitle{Bland - Altman Plot}

\begin{description}
  \item[Repeatability] - whereby a method agrees with itself
  \item[Three variability tests] - Roy demonstrates 3 formal tests for variability
  \item[Correlation] There should be an correlation of 0.82.
  \item[Numeric data - Continuous numeric data] Data which can assume an infinite number of values (including decimal places)
  between any two given values \textit{e.g.} height.
  \item[Numeric data - Discrete numeric data] Data that can only have a finite number of numeric values (whole numbers only) \textit{e.g.}
  family size.
\end{description}

}
\end{document}
