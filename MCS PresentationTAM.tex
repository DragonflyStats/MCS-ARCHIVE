
% sample.tex
\documentclass[pdf,azure,slideColor,colorBG]{prosper}
\begin{document}
	\section{The Technology Acceptance Model}
	Davis (1989) proposes the TAM model, which suggests an hypothesis as to why users may adopt particular technologies, and not others. 
	According to this theory, when users are presented with a new 
	technology, two important factors will influence their decision about how and when they will adopt it.
	\begin{description}
		\item[Perceived usefulness (PU)] - This was defined by Fred Davis as "the degree to which a person believes that using a particular system would enhance his or her job performance".
		\item[Perceived ease-of-use (PEOU)] - Davis defined this as "the degree to which a person believes that using a particular system would be free from effort" 
	\end{description}
\end{frame}
%============================================================================ %
\begin{frame}
\frametitle{The Technology Acceptance Model}	
	Davis's explanations of these term can be rephrased for application to statistical analysis. 
	Perceived Use could refer to the degree to which an user would deem a particular statistical method would properly establish the results of an analaysis. In the case of method comparison studies, proper indication of agreement, or lack thereof.
\end{frame}
%============================================================================ %
\begin{frame}
	\frametitle{The Technology Acceptance Model}	
	
	Perceived ease-of-use requires only applying the context of a satistical problem. A very modest statistical skill set is the only prerequistive for constructing a Bland-Altman plot, and computing limits of agreement. The main building blocks 
	are simple descriptive, statistics and a knowledge of the normal distribution. These are topics that feature in almost every undergraduate statistics courses. Furthermore \citet{kikozak2014including} recommends including the Bland-Altman method itself in undergraduate teaching.
	%---------------------------------------------%
	
	In short, the user perceives the Bland-Altman methodology to be an easy-to-implement technique, that will properly address the question of agreement.
\end{frame}
%============================================================================ %
\begin{frame}
	\frametitle{The Technology Acceptance Model}	
	Conversely the Survival plot is a derivative of the Kaplan-Meier Curve, a non-parametric graphical technique that features in Survival Analysis. This subject area is a well known domain of statistics, but would be encountered 
	on curriculums of specialist courses. 
\end{frame}
%============================================================================ %
\begin{frame}
	\frametitle{The Technology Acceptance Model}	
	The Mountain Plot is formally called the empirical folder cumulative distribution plot. While not particularly hard to render, the procedure is not straight-forward for the casual user. Currently there is only one software implementation, \textbf{\textit{medcalc.be}} toolkit.
	%\newpage
	%\section{Agreement Indices}
	%\citet{Barnhart} provided an overview of several agreement
	%indices, including the limits of agreement. Other approaches, such
	%as mean squared deviation, the tolerance deviation index and
	%coverage probability are also discussed.
\end{frame}
%============================================================================ %
\begin{frame}
	\frametitle{The Technology Acceptance Model}
	\section{Survival-Agreement Plot}
	A graphical technique for method comparison studies, that is entirely different to the Bland-Altman plot, was proposed by \citet{luiz}. This approach, known as the survival-agreement plot, is used to determine the degree of agreement using the Kaplan-Meier method, a well known graphical technique in the area of Survival Analysis. Furthermore \citet{luiz} propose that commonly used survival analysis techniques should complement this method,\textit{ providing a new analytical insight
		for agreement}. Two survival-agreement plots are used to detect the bias between to measurements of the same variable. The presence of inter-method bias is tested with the log-rank test, and its magnitude with Cox regression.
	
	%% TOLERANCE - REWRITE THIS
	
	The degree of agreement (or disagreement) of a measure is expressed as a function of several limits of tolerance, using the Kaplan-Meier method, where the failures occur exactly at absolute values of the differences between the two methods of measurement. 
	
	According to Luiz et al, the survival-agreement plot is a step function of a typical survival analysis without censored data, where the Y axis represents the proportion of discordant cases. This is equivalent to a step function where the X axis represents the absolute  observed differences and the Y axis is the proportion of the cases with at least the observed 
	difference ($x_i$). 
	
	% % PREVALENCE
	% % Implementation
	
	
	%============================================================================================================ %
	
	
	
	
	% MCS Mountain Plot Notebook
	
	\section{Mountain Plot} Krouwer and Monti have proposed a folded empirical cumulative distribution plot, otherwise known as a Mountain plot.
	
	They argue that it is suitable for detecting large, infrequent errors. This is a non-parametric method that can be used as a complement with the Bland Altman plot.  Mountain plots are created by computing a percentile
	for each ranked difference between a new method and a reference method. (Folded plots are so called because of the following transformation is performed for all percentiles above 50: percentile = 100 - percentile.) These percentiles are then plotted against the differences between the two methods.
	
	Krouwer and Monti argue that the mountain plot offers some following advantages. It is easier to find the central $95\%$ of the data, even when the data are not normally distributed. Also, comparison on different distributions can be performed with ease.
	
	
	
	
	
\end{document}
