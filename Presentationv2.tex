%01) What is a method comparison study
%02) Agreement
%03) Bland Altman plot (Intermethod Bias mention Correlation)
%04) Limits of Agreement
%05) Repeatability
%06) PEFR Data
%07) scatterplot of PEFR   [DONE]
%08) BA of PEFR             [DONE]
%09) Repeated measurements
%10) JSR data
%%--------------------------------------------------------------%
%11) LMEs
%12) VC Structures
%13) LMEs Repeated measurements (hamlett)
%14) Roys methods (Three Criteria)
%15) Variability Tests


\title[MCS]{Diagnostic measures for the analysis of method comparison
studies using the linear mixed effects model
 \\ {\normalsize Transfer Report Presentation}}
\author[Kevin O'Brien]{Kevin O'Brien \\ {\scriptsize kevin.obrien@ul.ie}}
\date{November 2010}
\institute[Maths \& Stats]{Dept. of Mathematics \& Statistics, \\ University \textit{of} Limerick}

\renewcommand{\arraystretch}{1.5}

\begin{document}

\begin{frame}
\titlepage
\end{frame}

\section{Method Comparison Studies}
\frame{
Method comparison studies:
\begin{itemize}
  \item Method comparison studies are used to assess the relative agreement between two methods that measure the same variable.\item Two methods of measurement giving consistently similar results, with the same level of precision are considered to be in agreement.
      \item Commonly a new method is compared to one currently in use to see whether their measurements are indeed comparable, and whether these two methods can be used interchangeably.\item Inter-method bias can be described as the tendency for one method to give a measurement higher than the counterpart method. The presence of inter-method bias indicates lack of agreement.
\item Importance of this area is reflected by the 18,860 citations of Bland and Altman's 1986 paper in the Lancet.
\end{itemize}

\vspace{0.2cm}
%Agreement between two methods of clinical measurement can be quantified using the differences between observations made using the two methods on the same subjects (Bland and Altman 1999).\\
}

%-------------------------------------------------------------------------------------%
\frame{\frametitle{Bland-Altman plots}
\begin{itemize} \item (Bland and Altman, 1986 and 1999) noted the inappropriate use of correlation for method comparison, and developed a simple graphical approach , known as a Bland-Altman plot, to compare two measurements methods. 
\item The approach requires the calculation and plotting the case-wise average and case-wise differences.
\item The mean of the case-wise differences  serves an estimate for the inter-method bias. A horizontal line to represent this is added to the plot.

\item Bland and Altman also advise the use of scatterplots as a preliminary analysis.
\end{itemize}
}

%
%%-------------------------------------------------------------------------------------%
%\frame{\frametitle{Inter-method bias}
%
%
%The mean of the casewise differences is the estimate for the intermethod bias.
%
%\begin{itemize}
% \item Constant Bias
% \item Proportional Bias
%\end{itemize}
%
% http://biopharminternational.findpharma.com/biopharm/data/articlestandard//biopharm/032002/7276/article.pdf
%
%Inter-method bias -----
%}
%-------------------------------------------------------------------------------------%
\frame{\frametitle{Limits of Agreement}

\begin{itemize}
\item Bland and Altman (1986) introduces a further element to their methodology, the limits of agreement.

\item The standard deviation of the case-wise differences is determined.

\item The 95\% limits of agreement are calculated as the inter-method bias plus and minus 1.96 standard deviation of casewise differences
    

\item Again, horizontal line are added to the Bland-Altman plot.

\item These limits are expected to contain the difference between measurements by the two methods for 95\% of pairs of future measurements on similar individuals.
    
    
\end{itemize}
}

%-------------------------------------------------------------------------------------%

\frame{\frametitle{Repeatability}
\begin{itemize}
\item Repeatability is relevant to the study of method comparison
because the repeatabilities of the two methods of measurement
limit the amount of agreement which is possible. \item If one method has
poor repeatability i.e. there is considerable variation in
repeated measurements on the same subject, the agreement between
the two methods is bound to be poor too.

\item Bland and Altman (1986) complemented their methodology with the coefficient of repeatability. The coefficient of repeatability is a precision measure which represents the value below which the absolute difference between two repeated measurements can be expected to lie with 95\% probability .
\end{itemize}
}



\section{PEFR Example}
%------------------------------------------------------------------------------------GOOD--%
\frame{\frametitle{Example: PEFR Data}
\begin{itemize}
\item This example uses data from Bland and Altman (1986).  Two measurements of peak
expiratory flow rate (PEFR) are compared. One of these measurements uses a ``Large" meter
and the other a ``Mini" meter.
\item The purpose of the procedure is to assess the agreement between both methods.
\end{itemize}
}

%------------------------------------------------------------------------------------GOOD--%

\frame{\frametitle{PEFR Data}

\begin{tabular}{|c|c|c|c|c|}
  \hline
 & Wright (1st)    &Wright (2nd)   & Mini (1st)   & Mini (2nd)\\
 \hline
Subject &    (l/min)    &(l/mi)&     (l/min)&    (l/min)\\
1&   494&    490    & 512    & 525 \\
2&   395&    397    & 430    & 415 \\
3&   516&    512    & 520    & 508 \\
4&   434&    401    &  428   & 444 \\
\vdots &   \vdots &    \vdots     &  \vdots   & \vdots \\
  \hline

\end{tabular}

The table above  tabulates the observed PEFR measured with both Wright peak flow and Mini peak flow meters, on 17 individuals. Each individual is measure twice with each meter. Bland and Altman only use the first measurements of each meter.
}
%------------------------------------------------------------------------------------GOOD--%
\frame{\frametitle{Scatter plot for PEFR data}
\begin{center}
\includegraphics[width=8cm]{wright-meter-scatterplot.pdf}
\end{center}
}
%------------------------------------------------------------------------------------GOOD--%
\frame{\frametitle{Bland-Altman plot for PEFR data}
\begin{center}
\includegraphics[width=8cm]{wright-meter-BAplot-CIs.pdf}
\end{center}
}

%----------------------------------------------------------------------------------------------%
\frame{\frametitle{Bland-Altman Plot : Multiple measurements}

\begin{itemize}
\item Method comparison studies will commonly use replicate measurements on each individual.
\item The Bland-Altman plot was specifically constructed for a single pair of measurements.\\
\item Bland and Altman (1999) extend the limits of agreement approach to data with repeated measurements. \\
\item The proposed approaches entail either treating measurements as a single measurement, or by considering the average of a set of repeated measures. 
\item Carstensen et al (2008) disapprove of this approach, and recommend using linear mixed effects models for assessing agreement in the presence of replicate measurements.
\end{itemize}

}
%----------------------------------------------------------------------------------------------%
\section{LME Models}
\begin{frame}\frametitle{Linear Mixed Effects model}

\begin{itemize}
\item A linear mixed effects (LME) model is a statistical model containing both fixed effects and random effects.
\item LME models are a generalization of the classical linear model, which contain fixed effects only.
\item
LME models are commonly formulated in the notation described in Laird and Ware (1982)
\[
y = X\beta + Zb + \epsilon
\label{LW}
\]
\item Of particular interest is the variance estimates associated with the random effects parameters\[
\mathrm{var}
\left(
              \begin{array}{c}
                b \\
                \epsilon \\
              \end{array}
            \right)
   =
\left(
         \begin{array}{cc}
           D & 0 \\
           0 & \Sigma \\
         \end{array}
       \right)
\]
\end{itemize}

\end{frame}
%----------------------------------------------------------------------------------------------%
\begin{frame}\frametitle{LME models in method comparison }
\begin{itemize}
\item Hamlett et al (2003) describe a formulation of the LME model to describe all responses of the $i-$th subject $(\boldsymbol{y}_{i})$, in the presence of two predictor variables.
\[
\boldsymbol{y}_{i} = \boldsymbol{X}_{i}\beta + \boldsymbol{Z}_{i}Z\boldsymbol{b}_{i} + \boldsymbol{\epsilon}_{i}
\]

\item Roy (2009) uses this approach to build an LME methodology for assessing agreement between two methods. The measurements are the responses, and both methods of measurement are the predictor variables. 
\item Roy's approach allows for specifying the structure of the between-subject variance $\boldsymbol{D}$ and the within-subject variance $\boldsymbol{\Sigma}$ of the measurement methods.
\end{itemize}
\end{frame}
%%----------------------------------------------------------------------------------------------%
%\frame{\frametitle{Replicate Measurements}
%\begin{itemize}
%\item In comparison, multiple observation are often made with each method on the same subject.
%\item These observations can be considered as replicate measurements if the observations with the same method on the same subject are conditionally independent and identically distributed.
%\end{itemize}
%}
%%----------------------------------------------------------------------------------------------%

\frame{\frametitle{Roy's agreement criteria}
Roy (2009) sets out three criteria for two methods to be considered in agreement. 
\begin{itemize}
\item Firstly that there be no significant bias. \item Second that there is no difference in the between-subject variabilities, and no significant difference in the within-subject variabilities. \item There should be no difference in the the overall variability of both methods. \item Roy demonstrates 3 formal tests for comparing the variabilities. \item These tests are realized by specifying different structures for the variance-covariance matrices.
    
\end{itemize}
}

%----------------------------------------------------------------------------------------------%

\frame{\frametitle{Variability tests}

The three variability tests are as follows
\begin{itemize}
\item Testing whether both methods have equal between-subject variances
\item Testing whether both methods have equal within-subject variances
\item Testing whether both methods have equal overall variability
\end{itemize}
 }

%----------------------------------------------------------------------------------------------%

\frame{\frametitle{Bland-Altman's JSR Data}

Roy includes a table from Bland and Altman's 1999 paper which shows a set of systolic blood pressure data from a study in which simultaneous measurements were made by each of two experienced observers (denoted J and R) using a sphygmomanometer and by a semi-automatic blood pressure monitor (denoted S). \bigskip Three sets of readings were made in quick succession.
 }
 
%%----------------------------------------------------------------------------------------------%
\begin{frame}[fragile]
\frametitle{Roy's Methodology  R implementation}

Roy's methodology requires the construction of four very similar LME models.
The first of Roy's candidate model (MCS1) can be implemented using the following code;\\
\begin{verbatim}
   MCS1 = lme(BP ~ method-1, data = dat,
   random = list(subject=pdSymm(~ method-1)),
   weights = varIdent(form=~1|method),
   correlation = corSymm(form=~1 | subject/obs), method="ML")
\end{verbatim}
The other three candidate models are constructed by using the same code, but exchanging the phrase ``CompSymm" for ``Symm" when appropriate.
\end{frame}

%----------------------------------------------------------------------------------------------%
\begin{frame}[fragile]\frametitle{Roy's likelihood ratio tests}
\textbf{Comparing Models:}\\
A form test may be implemented using a likelihood ration test, implemented in \texttt{R} with the \texttt{anova()}command.
\begin{verbatim}
> anova(MCS1,MCS2)
      Model df AIC    BIC    logLik  Test   L.Ratio  p-value
MCS1  1     8  4077.5 4111.3 -2030.7
MCS2  2     7  4075.6 4105.3 -2030.8 1 vs 2 0.15291  0.6958
>
\end{verbatim}
Further to this test, we fail to reject the model MCS1 in favour of MCS2.
Therefore we conclude both methods have the same between-subject variance.
\end{frame}






%----------------------------------------------------------------------------------------------%
\begin{frame}[fragile]\frametitle{Roy's fixed effects estimates}
\textbf{Fixed Effects estimates:}\\
The fixed effects estimates of any model can be used to determine the Inter-method bias. To compute the inter-method bias simple subtract one estimate from the other. \\However , to obtain explicit parameter estimates, such as p-value, a variation of the code, specifying a intercept, can be used;
\begin{verbatim}
   MCS1 = lme(BP ~ method, data = dat, ......
\end{verbatim}

\end{frame}

%%----------------------------------------------------------------------------------------------%
%\begin{frame}\frametitle{Variance Structures }
%\begin{itemize}
%\item The Identity variance structures is commonly used in LME models.
%\[\left(
%        \begin{array}{cc} \sigma^2_{1} &  \sigma_{12} \\   \sigma_{21} &  \sigma^2_{2} \\ \end{array}
%       \right)
%\]
%\end{itemize}
%% \begin{array}{cc} \sigma^2_{1} &  \sigma_{12} \\   \sigma_{21} &  \sigma^2_{2} \\ \end{array}
%Identity:  Diagonal terms are equal. Off Diagonal terms are zero.
%\\Compound Symmetry: Diagonal terms are equal. Otherwise unconstrained.
%\\Symmetric (also known as unstructured): Positive definite symmetric matrix.
%
%
%\end{frame}
%----------------------------------------------------------------------------------------------%
\begin{frame}\frametitle{Limits of Agreement}
\textbf{General LME models:}\\
Bendix Carstensen et al demonstrate a method of computing the limits of agreement using simple LME models. However, if the observations are assumed to be linked, an extra interaction term must be added to Carstensen LME model.\\
\vspace{0.2cm}
\textbf{Roy's LME models:}\\
Roy's LME models can also be used to compute the limits of agreement. The formulation of the model already accounts for linkage, so no additional terms are necessary.
\vspace{0.1cm}Limits of agreement are not given explicitly in the code output, but a simple \texttt{R} function can be used to construct them.
\end{frame}








%----------------------------------------------------------------------------------------------%
\section{Model Diagnostics}
\begin{frame}\frametitle{LME Diagnostics for LME models}

\begin{itemize}
\item Cook and Weisberg (1982) developed case deletion model diagnostics, such as Cook's distance for linear models.
\item Christensen, Pearson and Johnson (1992) examined case deletion diagnostics, in particular the LME equivalent
of Cook’s distance, for assessing influential observations in LME models.

\item Haslett and Hayes (1998) presented a general theory for residuals in the general linear model (GLM) framework, in which marginal and conditional residuals are described.
    
%\item Considering LMEs as augmented GLMs, it is proposed to apply Haslett and Hayes's approach to method comparison studies.
\end{itemize}



\end{frame}

%----------------------------------------------------------------------------------------------%

\begin{frame}\frametitle{Henderson Equation}
Henderson's Equation for estimating LME parameters can be formulated as
\[
\left(\begin{array}{cc}
  X^\prime\Sigma^{-1}X & X^\prime\Sigma^{-1}Z
  \\
  Z^\prime\Sigma^{-1}X & X^\prime\Sigma^{-1}X + D^{-1}
  \end{array}\right)
\left(\begin{array}{c}
    \beta \\
  b
  \end{array}\right)
  =
\left(\begin{array}{c}
  X^\prime\Sigma^{-1}y \\
  Z^\prime\Sigma^{-1}y
  \end{array}\right).\]


Henderson's equations can be rewritten as a GLM model in form $( T^\prime W^{-1} T ) \delta = T^\prime W^{-1} y_{a} $ using
\[
\delta = \left(\begin{array}{c}\beta \\ b \end{array}\right),
\ y_{a} = \left(\begin{array}{c}
  y \cr \psi
  \end{array}\right),
\ T = \left(\begin{array}{cc}
  X & Z  \\
  0 & I
  \end{array}\right),
\ \textrm{and} \ W = \left(\begin{array}{cc}
  \Sigma & 0  \cr
  0 &  D \end{array}\right),
\]
 
($\psi$ represents ``quasi-data", a construct used to analyze GLMs.)
 
\end{frame}

%----------------------------------------------------------------------------------------------%

\begin{frame}\frametitle{LMEs as Augmented GLMs}

\begin{itemize}\item This augmented GLM formulation allows the analyses developed by Haslett and Hayes (1998) to be used on LME models. \item It is proposed to investigate diagnostic measures, such as residuals Mahalanobis distances, for validating method comparison studies models.\\
\end{itemize}



\end{frame}


%-------------------------------------------------------------------------------------%

\section{Bibliography}
\begin{frame}\frametitle{Bibliography}
\begin{itemize}
\item D.G Altman and J.M. Bland (1983) - Measuring agreement in method comparison studies. ( \textit{Statistical Methods in Medical Research} ; \textbf{8}:135--160.)
\item J.M. Bland  and D.G Altman  (1986) - Measuring agreement in method comparison studies. ( \textit{Statistical Methods in Medical Research} ; \textbf{8}:135--160.)
\item J.M. Bland  and D.G Altman  (1999) - Measuring agreement in method comparison studies. ( \textit{Statistical Methods in Medical Research} ; \textbf{8}:135--160.)
\item N.M. Laird and J.H Ware (1982) - Random-Effects Models for Longitudinal Data, ( \textit{Biometrics}, \textbf{38}: 963–974.)
\end{itemize}
\end{frame}
%----------------------------------------------------------------------------------------------%
\begin{frame}\frametitle{Bibliography (contd)}
\begin{itemize}
\item Hamlett et al (1983) - Measuring agreement in method comparison studies. ( \textit{Statistical Methods in Medical Research} ; \textbf{8}:135--160.)
\item Carstensen   (2008) - Measuring agreement in method comparison studies. ( \textit{Statistical Methods in Medical Research} ; \textbf{8}:135--160.)
\item Anuradha Roy (2009) - Random-Effects Models for Longitudinal Data, ( \textit{Biometrics}, \textbf{38}: 963–974.)
\end{itemize}
\end{frame}


%----------------------------------------------------------------------------------------------%









\end{document}
